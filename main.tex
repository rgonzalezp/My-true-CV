%%%%%%%%%%%%%%%%%%%%%%%%%%%%%%%%%%%%%%%%%
% "ModernCV" CV and Cover Letter
% LaTeX Template
% Version 1.11 (19/6/14)
%
% This template has been downloaded from:
% http://www.LaTeXTemplates.com
%
%
% License:
% CC BY-NC-SA 3.0 (http://creativecommons.org/licenses/by-nc-sa/3.0/)
%
% Important note:
% This template requires the moderncv.cls and .sty files to be in the same 
% directory as this .tex file. These files provide the resume style and themes 
% used for structuring the document.
%
%%%%%%%%%%%%%%%%%%%%%%%%%%%%%%%%%%%%%%%%%

%---------------------------------------------------------------------https://www.overleaf.com/project/5dbf222255799b000143f25f-------------------
%	PACKAGES AND OTHER DOCUMENT CONFIGURATIONS
%----------------------------------------------------------------------------------------


\documentclass[11pt,a4paper,sans]{moderncv} % Font sizes: 10, 11, or 12; paper sizes: a4paper, letterpaper, a5paper, legalpaper, executivepaper or landscape; font families: sans or roman


\moderncvstyle{classic} % CV theme - options include: 'casual' (default), 'classic', 'oldstyle' and 'banking'
\moderncvcolor{blue} % CV color - options include: 'blue' (default), 'orange', 'green', 'red', 'purple', 'grey' and 'black'
\usepackage{xcolor}
\usepackage{lipsum} % Used for inserting dummy 'Lorem ipsum' text into the template
\usepackage[scale=0.8]{geometry} % Reduce document margins
%\setlength{\hintscolumnwidth}{3cm} % Uncomment to change the width of the dates column
\setlength{\makecvtitlenamewidth}{10cm} % For the 'classic' style, uncomment to adjust the width of the space allocated to your name


%----------------------------------------------------------------------------------------
%	NAME AND CONTACT INFORMATION SECTION
%----------------------------------------------------------------------------------------

\firstname{Ricardo E. Gonzalez Penuela } % Your first name


% All information in this block is optional, comment out any lines you don't need
\title{Curriculum Vitae/Resume}

\email{reg258@cornell.edu}
\homepage{https://rgonzalezp.github.io/} % The first argument is the url for the clickable link, the second argument is the url displayed in the template - this allows special characters to be displayed such as the tilde in this example
%\photo[70pt][0.4pt]{pictures/House} % The first bracket is the picture height, the second is the thickness of the frame around the picture (0pt for no frame)
%\quote{"A witty and playful quotation" - John Smith}
%----------------------------------------------------------------------------------------

\begin{document}

\makecvtitle % Print the CV title
%----------------------------------------------------------------------------------------
%	EDUCATION SECTION
%----------------------------------------------------------------------------------------

\section{Education}

\cventry{2021.1--Present}{Doctor of Philosophy (Ph.D.) in Information Science}{}{}{\textit{Information Science Department, \link[Cornell Tech, Cornell University]{https://www.tech.cornell.edu/}, New York, United States }}{}

\cventry{2015.1--2019.12}{B. Eng. in Systems and Computer Engineering}{}{}{\textit{Department of Systems and Computer Engineering, \link[Universidad de Los Andes]{https://sistemas.uniandes.edu.co/en/}, Bogota, Colombia }}{}




%----------------------------------------------------------------------------------------
%	Research Work and Publications SECTION
%----------------------------------------------------------------------------------------

\section{Research Work and Publications}
\cvitem{2022.10}{\textit{\textbf{Hands-On: Using Gestures to Control Descriptions of a Virtual Environment for People with Visual Impairments}:  Paper in submission \textbf{UIST2022}}}
\cvitem{}{\textit{\textbf{Ricardo E. Gonzalez Penuela}, Wren Poremba, Christina Trice, \& Shiri Azenkot. While researchers have explored how to make navigation and object perception more accessible in VR, none have offered a natural way to request descriptions of objects, nor control the flow of auditory information. We present a haptic glove that PVI can use to request object descriptions with their hands through hand gestures.}}

\cvitem{2022.10}{\textit{\textbf{Uncovering Visually Impaired Gamers' Preference for Spatial Awareness Tools Within Video Games}: Accepted paper to appear in \textbf{ASSETS `22} (26.5\% acceptance)}}
\cvitem{}{\textit{Authors: Vishnu Nair, Shao-en Ma, \textit{\textbf{Ricardo E. Gonzalez Penuela}}, Yicheng He, Karen Lin, Mason Hayes, Hannah Huddleston,   Matthew Donnelly, Brian A. Smith. We investigated four leading approaches to facilitate spatial awareness for visually impaired gamers within a 3D video game. We uncover what is the most important information for visually impaired gamers to gain spatial awareness, and how well our spatial awareness tools provide it.}}

\cvitem{2022.10}{\textit{\textbf{Understanding How People with Visual Impairments Take Selfies: Experiences and Challenges}:  Accepted poster to appear in \textbf{ASSETS `22} (59\% acceptance)}}
\cvitem{}{\textit{\textbf{Ricardo E. Gonzalez Penuela}, Paul Vermette, Zihan Yan, Cheng Zhang, Keith Vertanen, \& Shiri Azenkot. Selfies are a pervasive form of communication in social media. PVI want to participate in social media just like their sighted counterparts, so it is important to ensure that selfie-taking is accessible. We contribute design guidelines that researchers and designers can implement for creating accessible selfie-taking applications.}}


\cvitem{2021.10}{\textit{\textbf{Towards a Generalized Acoustic Minimap for Visually Impaired Gamers}: Demo in \textbf{UIST2021} (Overall acceptance 21\%)}}
\cvitem{}{\textit{Authors: Vishnu Nair, Shao-en Ma,Hannah Huddleston, Karen Lin, Mason Hayes, Matthew Donnelly, \textit{\textbf{Ricardo E. Gonzalez Penuela}}, Yicheng He, Brian A. Smith. We developed a prototype with four acoustic minimap techniques which would enable visually impaired gamers gain spatial awareness of a game environment.}}

\cvitem{2020.4}{\textit{\textbf{Molder: An Accessible Design Tool for Tactile Maps}: Paper in \textbf{CHI2020} (24.3\% acceptance)}}
\cvitem{}{\textit{Authors: Lei Shi, Yuhang Zhao, \textit{\textbf{Ricardo E. Gonzalez Penuela}}, Elizabeth Kupferstein, Shiri Azenkot. Molder is an accessible design tool for interactive tactile maps, an important type of printed materials that can help visually impaired students learn O\&M skills.}}

\cvitem{2019.10}{\textit{\textbf{Tactiled: Towards more and better tactile graphics using machine learning}: Poster in \textbf{ASSETS `19} (58\% acceptance)}}
\cvitem{}{\textit{\textbf{Gonzalez, R.}, Gonzalez, C., \& Guerra-Gomez, J. A. (2019). Tactiled. The 21st International ACM SIGACCESS Conference on Computers and Accessibility - ASSETS 2019. Presented at the The 21st International ACM SIGACCESS Conference.}}

\cvitem{2019.8}{\textit{\href{https://github.com/rgonzalezp/Markit}{\textcolor{blue}{Markit}}}}
\cvitem{}{\textit{Lei Shi, \& \textbf{Ricardo E. Gonzalez Penuela}. (2019, November 2). rgonzalezp/Markit: Markit V1.0 (Version V1.0). Zenodo. http://doi.org/10.5281/zenodo.3526177}}

%----------------------------------------------------------------------------------------
%	Research community involvement SECTION
%----------------------------------------------------------------------------------------

\section{Research Community Involvement}
\cvitem{}{\textit{\textbf{Community}}}
\cvitem{2021.1-- Present}{\textit{\href{https://xraccess.org/research/}{\textcolor{blue}{XR Access Research Network}}. Working closely together with  \href{https://xraccess.org/about/}{Professor \textcolor{blue}{Shiri Azenkot}}, Co-Founder of \href{https://xraccess.org/}{\textcolor{black}{XR Access}}, to run all activities related to the XR Access Research Network. This includes:}}
\cvitem{}{Organizing, recruiting speakers, and hosting the \href{https://xraccess.org/research/}{\textbf{XR Access Research Network Seminar}.} We have brought over 7 community leaders in the space of XR Accessibility to share their work.}
\cvitem{}{Mentoring in the \href{https://xraccess.org/reu/}{\textbf{REU site} (Research Experience for Undergrads)} program \textbf{hosted by the XR Access Research Network} and \textbf{funded by the NSF}, Summers of 2021, and 2022. At both opportunities, fully-funded undergrad students collaborated in projects that have become submissions/fully-published papers at significant conferences (ASSETS,CHI, UIST) or that are currently in submission.}
\cvitem{}{Recruiting speakers, and hosting the \href{https://www.youtube.com/watch?v=ZG0w6l4qRr4&list=PLIpr16Y-xnVlXThQPqkvnQuY8GyiDjgEn&index=7}{\textbf{Research to Practice Panel}} at the \href{https://xraccess.org/symposium/}{\textbf{XR Access Symposium 2022}}.}

\cvitem{}{\textit{\textbf{Peer Reviewing}}}
\cvitem{}{\textit{Reviewer at \href{https://mobilehci.acm.org/2022/}{\textcolor{black}{Mobile HCI 2022}}.}}

%----------------------------------------------------------------------------------------
%	Awards and Media SECTION
%----------------------------------------------------------------------------------------

\section{Awards and Media}
\cvitem{}{\textit{\textbf{Fellowships \& Grants}}}
\cvitem{Fall 2022-- Fall 2023}{\textit{Recipient  of the Digital Life Initiative (DLI) \textbf{Doctoral Fellowship}. \href{https://www.dli.tech.cornell.edu/join}{\textcolor{black}{``The Program engages students in systematic inquiry into ethical and political implications of the digital age. Such inquiry centers on issues – e.g. fairness, security, privacy, and accountability – that concern existing and emergent digital technologies.``}}}}
\cvitem{}{\textit{\textbf{Media}}}
\cvitem{2019.6}{\textit{Note on \textbf{Assistive Technologies} work in Colombian newspaper \textbf{El Espectador}: \href{https://www.elespectador.com/noticias/ciencia/ingenieros-colombianos-desarrollan-herramientas-para-ayudar-personas-ciegas-articulo-870825}{\textcolor{blue}{``Colombian engineers develop tools to help blind people``}}}}










%----------------------------------------------------------------------------------------
%	Teaching Job Experience SECTION
%----------------------------------------------------------------------------------------
\section{Teaching Job Experience}

\cvitem{Spring 2021} {\textit{Teacher assistant for class INFO 5305: \textbf{User Experience and User Research methods},
\textbf{Cornell Tech}, with Professor Shiri Azenkot}}

%----------------------------------------------------------------------------------------
%	Research Job Experience SECTION
%----------------------------------------------------------------------------------------
\section{Research Job Experience}

\cvitem{2019.5--2019.8}{\textit{\textbf{Research Assistant Intern} at \textbf{Cornell Tech} at the Enhancing Ability Lab (\textbf{Accessibility}) with \href{http://shiriazenkot.com/}{ Professor \textcolor{blue}{Shiri Azenkot}, ``Summer Undergraduate \textbf{Research Fellowship}``}}}
\cvitem{}{}


\cvitem{2019.8--2020.12}{\textit{\textbf{Research Assistant} at \textbf{Cornell Tech} at the Enhancing Ability Lab (\textbf{Accessibility}) with \href{http://shiriazenkot.com/}{Professor \textcolor{blue}{Shiri Azenkot}}}}

\cvitem{}{\textit{Presented \textbf{Molder} demo at \textbf{ASSETS `19} on behalf of \href{http://shiriazenkot.com/}{Professor \textcolor{blue}{Shiri}} and \href{https://scholar.google.com/citations?user=BxSTxa0AAAAJ&hl=en}{\textcolor{blue}{Lei Shi}}, see more in the Research Work section}}



%----------------------------------------------------------------------------------------
%	Leadership and Volunteering SECTION
%----------------------------------------------------------------------------------------

% To remove the cover letter, comment out this entire block

%\clearpage

%\recipient{HR Department}{Corporation\\123 Pleasant Lane\\12345 City, State} % Letter recipient
%\date{\today} % Letter date
%\opening{Dear Sir or Madam,} % Opening greeting
%\closing{Sincerely yours,} % Closing phrase
%\enclosure[Attached]{curriculum vit\ae{}} % List of enclosed documents

%\makelettertitle % Print letter title

%\lipsum[1-3] % Dummy text

%\makeletterclosing % Print letter signature

%----------------------------------------------------------------------------------------

\end{document}



